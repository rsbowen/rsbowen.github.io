As a math undergraduates, I took a course in the history of math. I was
surprised to find just how close they were to a modern understanding of
analysis and calculus. There are a few examples: Archimedes computes the limit
of a geometric series; the method of exhaustion looks like a limit if you
squint; a few places in greek mathematics look suspiciously like integrals.
This cool fact is relatively well known but I'm not aware of anywhere that
collects them in a simple form on the internet -- that is the purpose of this
project. Here I will collect a way in which the greeks seem to my eye to come
close to a modern foundational definition of the reals.

\section{Eudoxus and Dedekind}
Eudoxus of Cnidus was a 4th- and 3rd- century BC Greek mathematician. After the
Pythagoreans had discovered the incomensurability of the side of a square with
the diagonal, Eudoxus made an important contribution by giving a way to determine
the equality of ratios of magnitudes in terms of conditions only on natural
numbers. Richard Dedekind was a 19th- and 20th- century German mathematician
famous for the ``Dedekind cut'', a formal construction of the reals from the
rationals.

Much of early Greek mathematical thought is predicated on the supremacy of
``number'', which we might now call the positive integers. The importance of
geometry to the Greeks can't be understated. In the context of lengths, there
is no obvious ``base unit'' (like the inch), so a single line can't be said to
have rational or irrational length. It is only in comparing lengths that the
idea of number becomes important. 

To compare two ratios we need to have four lines -- in two pairs. For clarity
let's call them the red lines and the blue lines. We're interested to know if
the red lines are in the same ratio (long to short, say) as the blue. If the
red pair are comensurable, and so are the blue, it's easy to use Euclid's
algorithm to verify that the ratio is the same, even if the left lines and the
right lines are not comensurable. But Euclid's algorithm won't work on
incomensurable pairs.

\subsection{Eudoxus}
Eudoxus gives a way again to say whether two sets of lines have the same ratio,
even if they are not comensurable. Euclid states it as follows (cite -- RSB):
\begin{quote}
Magnitudes are said to be in the same ratio, the first to the second and the
third to the fourth, when, if any equimultiples whatever be taken of the first
and third, and any equimultiples whatever of hte second and fourth, the former
equimultiples alike exceed, are alike equal to, or alike fall short of, the
latter equimultiples respectively takein in the corresponding order
\end{quote}
That is, if I want to know if my red lines are in the same ratio as my blue
lines, I should pick any integer whatsoever, and repeat end-to-end the short
lines Then, I should pick any other integer, and repeat end-to-end the long
lines that many times. If no matter which integers I chose, the relationship
(shorter, longer, or the same length) between the red repeated-short line and
red repeated-long line is still the same as that between the blue
repeated-short line and the blue repeated-long line, then the ratios are the
same. In contemporary notation, we might say:
\[ \frac{a}{b} = \frac{c}{d} \leftrightarrow \left [ \forall x \forall y xa \geq yb
\leftrightarrow xc \geq yd \right ] \]
for \emph{integers} $x$ and $y$. Here is a short proof of this fact using
contemporary language. Forward direction: let $x$ and $y$ be arbitrary
integers. Then:
\begin{align*}
\frac{a}{b} &= \frac{c}{d}\\
\frac{xa}{yb} &= \frac{xc}{yd}
\end{align*}
This identity is useful in the following chain of coimplications:
\[xa \geq yb \leftrightarrow \frac{xa}{yb} \geq 1 \leftrightarrow 
\frac{xc}{yd} \geq 1 \leftrightarrow xc \geq yd\]
This completes the forward direction. The reverse can be gotten by
contradiction.  Suppose that
\[\forall x \forall y xa \geq yb \leftrightarrow xc \geq yd\]
but that, without loss of generality, $a/b < c/d$. Then there
is some rational $q=m/n$, where $m$ and $n$ are integers, which lies between
them: $a/b < m/n < c/d$. We may derive:
\[ a/b < m/n \rightarrow na < mb.\]
And simultaneously
\[ c/d > m/n \rightarrow nc > md.\]
This is a contradicition to our supposition by letting $x=n$ and $y=m$.

\subsection{Eudoxus and Dedekind}
Dedekind constructs the real numbers as subsets $S$ of the rationals subject to
the property that $q \in S$ and $p<q$ implies $p \in S$, i.e., they're closed
downward. We can identify the real with the obvious limit point; for example,
$\sqrt{2}$ is identified with the set of all rationals whose square is less
than 2.

The Eudoxus condition does not explicitly give an ``object'' which we might consider a real
number. This is not needed, because
philosophically the Greeks didn't consider the reals to be first-class objects,
only considering them to be ratios of first-class magnitudes. But suppose we want to look
at Eudoxus through this ontological lens. For any ratio, we might think of an infinite table:
each row corresponds to to an integer by which we will repeat the shorter line;
each column to an integer by which we will repeat the longer line. In each cell of the table we place ``$<$'' or ``$\geq$'',
according the new relationship (so in the $1,1$ cell we place $<$). This we
identify with the ratio. Eudoxus' condition that two ratios are equal is
exactly that the two corresponding tables are identical.
The unification of these two formulations is straightforward: if and only if the
cell in row $m$ and column $n$ says ``$<$'' in the Eudoxus table, then add $m/n$
to Dedekind's set. For example, consider two lines in the ratio $1:\sqrt{2}$.
Any value where
\[1\cdot m < \sqrt{2} \cdot n\]
must have
\[m^2 < 2n^2\]
and thus
\[(m/n)^2 < 2\]
which is exactly the condition for $m/n$ to be in $\sqrt{2}$'s Dedekind cut.
The Eudoxus step of repeating the lines $m$ and $n$ times is exactly the same
as the condition that the ratio is at least $m/n$, and the Eudoxus requirement
of comparing *all* $m$ and $n$ is identical to the closed-downward condition.
Once we introduce the idea that we should have representative objects built out
of (infinitely many) integer- or rational- like objects which we identify with
reals (a philosophical idea, perhaps), the actually mathematics follows quickly
-- discovering the above equivalence is not difficult.
\end{document}
